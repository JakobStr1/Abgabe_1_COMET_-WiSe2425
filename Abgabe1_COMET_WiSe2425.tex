\documentclass{article}
\usepackage{amsmath} 
\usepackage{amssymb}
\usepackage{hyperref}
\usepackage{graphicx}
\usepackage{parskip}
%\usepackage[utf8]{inputenc}
%\usepackage[T1]{fontenc}
\usepackage[ngerman]{babel}
\usepackage{natbib}

\title{Abgabe 1 für Computergestützte Methoden}
\author{Gruppe: 70: \\Jakob Striegel {\small4351490} \\Marvin Mundhenke {\small4112977} \\Xhoi Dragoj {\small3983991}}
\date{01.12.2024}
\begin{document}
\maketitle
\tableofcontents
\newpage
\section{Der zentrale Grenzwertsatz}
Der zentrale Grenzwertsatz (ZGS) ist ein fundamentales Resultat der Wahrscheinlichkeitstheorie, das die Verteilung von Summen unabhängiger, identisch verteilter ($i.i.d.$) Zufallsvariablen (ZV) beschreibt. Er besagt, dass unter bestimmten Voraussetzungen die Summe einer großen Anzahl solcher ZV annährend normalverteilt ist, unabhängig von der Verteilung der einzelnen ZV. Dies ist besonders nützlich, da die Normalverteilung gut untersucht und mathematisch handhabbar ist.



\subsection{Aussage}
Sei $ X_{1} ,X_{2},..., X_{n} $ eine Folge von $i.i.d.$ ZV mit dem Erwartungswert $\mu$  = $\mathbb{E}(X_{i})$ 
und der Varianz $\sigma^2$ = Var($X_{i}$), wobei  0 $< \sigma^2 <$ $ \infty$ gelte. Dann konvergiert
die standardisierte Summe $Z_{n}$  dieser ZV für $n \to  \infty$  in Verteilung gegen eine
Standardnormalverteilung:\footnote[1]{Der zentrale Grenzwertsatz hat verschiedene Verallgemeinerungen. Eine davon ist der \textbf{Lindeberg-Feller-Zentrale-Grenzwertsatz} \citep[Seite 328]{klenke}, der schwächere Bedingungen an die Unabhängigkeit und die identische Verteilung der ZV stellt.}

\begin{equation}
    \label{Formel1}
    Z_{n}= \frac{\sum_{i=1}^{n} X_{i}-n\mu}{\sigma \sqrt{n}} \overset{d}{\to} \mathcal{N}(0,1).
\end{equation}
Das bedeutet, dass für große $n$ die Summe der ZV näherungsweise normalverteilt ist mit Erwartungswert $n\mu$ und die Varianz $n\sigma^2$:

\begin{equation}
    \label{Formel2}
    \sum_{i=1}^{n} X_{i} \sim \mathcal{N}(n\mu,n\sigma^2).
\end{equation}



\subsection{Erklärung der Standardisierung}
Um die Summe der ZV in eine Standardnormalverteilung zu transformieren, substrahiert man den Erwartungswert $n\mu$ und teilt durch die Standardabweichung $\sigma\sqrt{n}$. Dies führt zu der obigen Formel \eqref{Formel1}. Die Darstellung \eqref{Formel2} ist für $n \to \infty$ nicht wohldefiniert.
\subsection{Anwendungen}
Der ZGS wird in vielen Bereichen der Statistik und der Wahrscheinlichkeitstheorie angewendet. Typische Beispiele sind:
\begin{itemize}
    \item Durchführung von Hypothesentests.
    \item Berechnung von Konfidenzintervallen für Stichprobenmittelwerte.
\end{itemize}



\newpage
\section{Bearbeitung zur Aufgabe 1}
\subsection{Datenverarbeitung}
\begin{itemize}
    \item CSV-Datei herunterladen und in Excel öffnen
    \item Wir finden eine nicht standardisierte Tabelle für alle Standorte von Fahrradverleihen vor. Aufgebaut ist die Tabelle ohne eigene Spalten. Sämtliche Attribute werden hier durch Kommata getrennt. Die Tabelle beinhaltet 100 Gruppen. Jeder dieser 100 Gruppen wurde genau ein Standort eines Fahrradverleihs zugeordnet. Zu diesen Verleihen finden sich in der Tabelle mehrere Messreihen inklusive beispielsweise der zugehörigen Messwerte für Temperaturen oder Ähnliches wieder.
    \item Unter \dq Daten\dq{} und dann \dq Text zu Spalten\dq{} stellt man nach Komma trennen ein.
    \item Dann \dq group\dq{} filtern und Gruppe 70 aussuchen.
    \item Befehl \dq MAX(J:J)\dq, um die höchste mittlere Temperatur zu finden: ${83}^\circ F$ und umgerechnet ${28.33}^\circ C$
    \item Umwandeln in Excel ohne manuelle Umrechnung ebenfalls möglich: \\UMWANDELN(MAX(J:J);\dq F\dq;\dq C\dq).
\end{itemize}
\newpage
\subsection{Datenhaltung}
sqlite3.exe gemeinsam mit der CSV-Datei, welche unsere Daten enthält, in einen neuen Ordner kopieren und  sqlite3.exe ausführen. Folgende Befehle der Reihe nach eingeben:
\begin{itemize}
    \item .open verleih.db
    \item create table FVerleih (\\ \dq group\dq{} INTEGER, \\station TEXT, \\date TEXT, \\day\_of\_year INTEGER, \\day\_of\_week INTEGER, \\month\_of\_year INTEGER, \\precipitation TEXT, \\windspeed TEXT, \\min\_temperature INTEGER, \\average\_temperature INTEGER, \\max\_temperature INTEGER, \\count INTEGER);
    \item .separator ,
    \item .import bike.csv FVerleih
    \item select max(average\_temperature) from FVerleih where \dq group\dq=70 and average\_temperature between -1 and 99;
\end{itemize} 
Oder alternative und weniger schreibaufwendige Methode:
\begin{itemize}
    \item .open fahrradverleih.db
    \item .headers on
    \item .mode csv
    \item .import bike.csv bike
    \item .schema
    \item .mode columns
    \item select max(average\_temperature) from bike where \dq group\dq=70 and \\average\_temperature between -1 and 99;
\end{itemize}
Das Ergebnis des select-Befehls ist dann ${83}^\circ F$ und umgerechnet wieder ${28.33}^\circ C$.
\subsection{GitHub}
\url{https://github.com/JakobStr1/Abgabe_1_COMET_-WiSe2425}


\newpage
\bibliographystyle{plain} 
\bibliography{references}


\end{document}

% \